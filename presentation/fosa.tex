\section{FOSA}

\subsection{Entstehung}
\begin{frame}
	\frametitle{Wie ist FOSA entstanden? \hfill{} \footnotesize{FOSA}}
\end{frame}

\begin{frame}
	\frametitle{Wer macht bei FOSA mit? \hfill{} \footnotesize{FOSA}}
\end{frame}

\subsection{Ziele}
\begin{frame}
	\frametitle{Wozu eigentlich FOSA? \hfill{} \footnotesize{FOSA}}
	Wir möchten eine \textbf{freie} Sammlung von 
	Formelbüchern erstellen, welche auf die Inhalte der HSLU-T\&A
	abgestimmt sind.
	\vfill{}
	\begin{exampleblock}{Die vier Freiheiten 
		\hfill{} 
		\footnotesize{R. Stallman}}
		\begin{itemize}
			\item[F0] freies Benutzen des Originals 
				\hfill{} \footnotesize{"`run"'}
				\normalsize
			\item[F1] freies Teilen des Originals
				\hfill{} \footnotesize{"`redistribute"'}
				\normalsize
			\item[F2] freies Anpassen 
				\hfill{} \footnotesize{"`change"'}
				\normalsize
			\item[F3] freies Teilen des Geänderten 
				\hfill{} \footnotesize{"`release"'}
				\normalsize
		\end{itemize}
	\end{exampleblock}
\end{frame}

\begin{frame}
	\frametitle{Wieso ist uns FreeSoftware wichtig? 
		\hfill{} 
		\footnotesize{FOSA}}
	\textit{"`Freiheit wird in der Regel verstanden als die 
	Möglichkeit, ohne Zwang zwischen verschiedenen Möglichkeiten 
	auswählen und entscheiden zu können."'} 
	
	\hfill{} --- (Wikipedia)
	\vfill{}
	\textbf{Alles was obige Definition oder die vier Freiheiten
	verletzt, ist für uns schlicht keine Option!}
	\vfill{}
	\begin{exampleblock}{Unsere Freiheiten sind unantastbar dank \dots}
		\begin{itemize}
			\item Free Software 
				\hfill{} 
				\footnotesize{Tools, Formate}
				\normalsize
			\item General Public License (Copyleft) 
				\hfill{}
				\footnotesize{gpl-violations.org}
				\normalsize
		\end{itemize}
	\end{exampleblock}
\end{frame}
